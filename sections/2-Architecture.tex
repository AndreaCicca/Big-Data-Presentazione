\section{Architecture}

%------------------------------------------------
\begin{frame}
	\frametitle{Optimistic concurrency control}
 In a system with \textbf{optimistic concurrency control}, transactions operate under the optimistic assumption that conflicts are rare. 
\vspace{0.5cm}

 Consequently, instead of applying locks or rigid coordination mechanisms to manage concurrency, transactions proceed independently, and conflict resolution is checked only at the end
\end{frame}

%------------------------------------------------
\begin{frame}
	\frametitle{Multiversion Concurrency Control (MVCC)}
  FoundationDB utilizes multiversion concurrency control as part of its optimistic approach. With MVCC, each transaction operates on a "version" of the data, reflecting a database state at a certain point in time. \vspace{0.5cm}
  
  Changes made by a transaction are not applied directly to existing data but are recorded as new versions. This allows multiple transactions to operate simultaneously without interfering with each other.
\end{frame}
%------------------------------------------------


\begin{frame}
    \frametitle{Architecture and Transaction Processing}
    \begin{columns}
        \begin{column}{0.5\textwidth}
            \begin{itemize}
                \item Control Plane
                \item Data Plane
            \end{itemize}
        \end{column}
        \begin{column}{0.5\textwidth}
            \centering
            \includegraphics[width=\textwidth]{img/2-Architecture/Architecture and transaction processing.png}
        \end{column}
    \end{columns}
\end{frame}


%------------------------------------------------

\begin{frame}
	\frametitle{Control plane P1}

The control plane is responsible for persisting critical metadata of the cluster for high availability.
\textbf{Cluster Controller} monitors all servers in the cluster and recruits 3 processes:
\begin{itemize}
    \item Sequencer (Data Plane)
    \item Data Distributor
    \item Ratekeeper
\end{itemize}

Coordinators form a Paxos group (consensus protocol) and elect the Cluster Controller.
\end{frame}

%------------------------------------------------


\begin{frame}
    \frametitle{Control plane P2}
    

\textbf{Data Distributor} is responsible for monitoring failures and balancing data among Storage Servers.\\
\textbf{Ratekeeper} provides overload protection for the cluster.
	
\end{frame}

%------------------------------------------------

\begin{frame}
	\frametitle{Data plane: Transaction system (TS) P1}
 
\begin{itemize}
    \item \textbf{Sequencer}: assigns a read and a commit version to each transaction
    \item \textbf{Proxies}: offer MVCC read versions to clients and orchestrate transaction commits
    \item \textbf{Resolvers}: check for conflicts among transactions (5 seconds of history)
\end{itemize}
	
\end{frame}

%------------------------------------------------

\begin{frame}
	\frametitle{Data plane: Transaction system (TS) P2}

FoundationDB uses \textbf{multiversion concurrency control} to provide transactionally isolated reads without locking data or blocking writes.\\
\textbf{Optimistic concurrency control} ensures that deadlocks are impossible and that slow or failing clients cannot interfere with the operation of the database.
	
\end{frame}

% %------------------------------------------------


\begin{frame}
	\frametitle{Data plane: Log system (LS)}

\textbf{Log Servers} act as replicated, distributed persistent queues, each queue storing WAL data for a Storage Server.
	
\end{frame}

% %------------------------------------------------


\begin{frame}
	\frametitle{Data plane: Storage system (SS)}
The Storage system consists of a number of \textbf{Storage Servers}, each
storing a set of data shards (contiguous key ranges),
and serving client reads. Storage Servers are the majority of processes in the system, and together they form a distributed B-tree.
	
\end{frame}

% %------------------------------------------------
\begin{frame}
	\frametitle{Scaling}
Processes are assigned different roles, scaling is just adding processes for each role. \\

\begin{itemize}
    \item Clients read from sharded StorageServers, so reads scale
linearly with the number of StorageServers. 
\item Clients read from sharded StorageServers, so reads scale
linearly with the number of StorageServers. 
\end{itemize}

Coordinators and control plane's singleton processes in the  like \textbf{Cluster Controller} and \textbf{Sequencer} only perform limited metadata operations.
	
\end{frame}
% %------------------------------------------------


\begin{frame}
	\frametitle{Transactions: Read}
\begin{enumerate}
    \item Client need to contact one of the Proxies to obtain a read version (timestamp).
    \item Proxy contacts the Sequencer it requests a read version from the Sequencer that is at least as large as all previously issued transaction commit versions.
    \item Proxy sends read version to client.
    \item Client reads from Storage Servers using the read version issued by the Proxy.
    \item Client can cache this information for the next read if needed.
\end{enumerate}

\end{frame}

% %------------------------------------------------
\begin{frame}
	\frametitle{Transactions: Writes}
\begin{enumerate}
    \item When the client decides to commit the transaction, it sends the transaction data to one of the Proxies (read and write that you have done in the transaction)
    \item Client waits for a response from the Proxy (commit or abort)

\end{enumerate}

\end{frame}



% %------------------------------------------------

\begin{frame}
	\frametitle{Transactions: Proxy commits}
\begin{enumerate}
    \item Contacting the Sequencer for commit version that surpasses any existing read versions or commit versions.
    \item The Sequencer chooses the commit version by advancing it at a rate of one million versions per second
    \item Transaction is forwarded to Resolvers that have to hanlde the optimistic concurrency control by checking for read-write conflicts.
    \item Transaction is forwarded to a set of designated Log Servers
    \item Proxy reports the committed version to the Sequencer, ensures that later transactions' read versions occur after this commit
    \item Proxy replies to the client
\end{enumerate}
\end{frame}

% %------------------------------------------------

\begin{frame}
	\frametitle{Transactions P3}

 FoundationDB also offers
 
 \begin{itemize}
     \item \textbf{Read-Only Transactions}: They occur at a specific read version, ensuring serializability, while leveraging Multi-Version Concurrency Control (MVCC) for efficiency
     \item \textbf{Snapshot Reads}: consistent view of the database at a specific point in time (no conflict in transactions)
 \end{itemize}

\end{frame}


% %------------------------------------------------


\begin{frame}
	\frametitle{Strict serializability}
\begin{itemize}
  \item FDB ensures strict serializability via Serializable Snapshot Isolation (SSI), combining OCC with MVCC.
  \item Transactions obtain read and commit versions from the Sequencer, ensuring a sequential order.
  \item Commit versions serve as Log Sequence Numbers (LSNs), preventing gaps between transactions.
  \item Proxies provide LSN and previous LSN to Resolvers and LogServers for transaction processing.
  \item Resolvers utilize lock-free conflict detection algorithms, admitting transactions only when all agree.
  \item The key space is divided among Resolvers for parallel conflict detection.
  \item FDB's Optimistic Concurrency Control design simplifies TS and SS interactions but results in wasted work for aborted transactions.
\end{itemize}

 \end{frame}

%------------------------------------------------


\begin{frame}
	\frametitle{Logging protocol ***}
    Test

\end{frame}

% %------------------------------------------------


\begin{frame}
	\frametitle{Transaction system recover}
    \begin{itemize}
    
        \item Traditional databese uses ARIES Recovery (checkpoint, redo, undo)
        \item in FD the recovery is purposely made very cheap—there, greatly simplifying design choice: redo log processing is the same as the normal log forward path
        \item The recovery system needs to find out the end of the redo log and the Storage Servers will asynchronously replay the log
        
\end{itemize}
\end{frame}

%------------------------------------------------
\begin{frame}
	\frametitle{After detecting a failure ***}
    \begin{itemize}
        \item The recovery process starts by
detecting a failure and recruiting a new transaction system
\item CC (Cluster Controller) reads the previous TS configuration from Coordinators and locks this information
to prevent another concurrent recovery
\item CC recovers previous TS system states, including information about older
Log Servers
\item CC recruits a new set of Sequencer, Proxies, Resolvers,
and Log Servers
\item need to ensure all transactions in the LS are durable and retrievable by
Storage Servers for recovery

    \end{itemize}

\end{frame}



%------------------------------------------------


\begin{frame}
	\frametitle{Replication: Metadata}
    \begin{itemize}
      \item System metadata of the control plane is stored on Coordinators using Disk Paxos (consensus protocol used for reliable data replication across multiple nodes optimized for disk storage)
      \item As long as a quorum (majority) of Coordinators are live, this metadata can be recovered.

\end{itemize}
\end{frame}

 %------------------------------------------------


\begin{frame}
	\frametitle{Replication: Log}
\begin{itemize}

      \item When a Proxy writes logs to LogServers, each sharded log record is synchronously replicated on \( k = f + 1 \) LogServers.
      \item Only when all \( k \) have replied with successful persistence can the Proxy send back the commit response to the client.
      \item Failure of a LogServer results in transaction system recovery.

\end{itemize}
\end{frame}

 %------------------------------------------------


\begin{frame}
	\frametitle{Replication: Storage}
\begin{itemize}

      \item Every shard (key range) is asynchronously replicated to \( k = f + 1 \) StorageServers, forming a "team".
      \item A StorageServer usually hosts multiple shards (key range) to evenly distribute data across many teams.
      \item Failure of a StorageServer triggers DataDistributor to move data from affected teams to healthy ones.

\end{itemize}
\end{frame}